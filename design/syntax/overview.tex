\section{Basic syntax}

The syntax of Amulet is primarily a mixture of OCaml and Haskell. It uses indentation to determine blocks of code.

\subsection{Identifiers and Naming Conventions}
Identifiers can be composed of most Unicode character. There are some restrictions:

\begin{itemize}
\item Identifiers cannot contain \texttt{[](){},;"\`{}\textquotesingle}.
\item Identifiers cannot start with a digit.
\item Identifiers cannot contain the ``Other'' and ``Separator''\cite{unicodeCats}\cite{unicodeData} character groups.
\item The symbols \texttt{+-/\textbackslash*.,:|} cannot be mixed with symbols not in that set.
\end{itemize}

You can also use double backticks to have a multi-word variable name: \texttt{\`{}\`{}This is a fancy variable name\`{}\`{}}.

A period can be used to access child variables in modules.

\begin{grammar}
<start char>  ::= a | b | c | ...

<later char>  ::= <start char> | 0 | 1 | 2 | ...

<ident>       ::= <start char> \{ <later char> \}
             \alt \{ `+' | `-' | ... \}

<name>        ::= <ident> \{ `.' <ident> \}
\end{grammar}

Amulet has several conventions for names:
\begin{itemize}
\item Modules, types and type constructors should be written using PascalCase.
\item Variables, function names, type parameters and record field names should be lowercase.
\end{itemize}

\subsection{Comments}
Comments come in two forms: line comments (comments which last from the delimiter until a line break) and block comments (comments which last until the terminating block).

Line comments are started with \texttt{;}.
Block comments are started with \texttt{(*} and finished with \texttt{*)}. These can be nested to allow easier commenting out of code.

You can document a binding using \texttt{;;} or \texttt{(** ... *)}. Documentation comment can be exported from a file to HTML with the contents rendered as Markdown.
