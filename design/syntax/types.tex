\section{Types}

\subsection{Primary types}
Types are composed of several key expressions:

\begin{grammar}
<type> ::= <name>
      \alt <type application>
      \alt <forall>
      \alt <constraint>
      \alt <type function>
      \alt <type tuple>
      \alt `(' <type> `)'
      \alt `_'
      \alt `[' <type> `]'
      \alt `[|' <type> `|]'
      \alt <record>
      \alt <row>
      \alt <union>
      \alt <unifies>

<var> ::= <ident>
     \alt `(' <ident> : <type> `)'

<type application> ::= <type> <type>
                  \alt <type> <name> <type>
                  \alt <type> \lit{\`{}} <name> \lit{\`{}} <type>

<forall>        ::= `forall' \{ <var> \} `.' <type>

<constraint>    ::= <type> `=>' <type>

<function type> ::= <type> `->' <type>

<tuple type>    ::= <type> \{ `*' <type> \}

<unifies>       ::= <type> \lit{$\sim$} <type>
\end{grammar}
The $type\ application$, $forall$, $constraint$ and $type\ function$ rules are right associative, whilst $unifies$ is non-associative.

\todo[inline]{What is the precedence of these?}

\subsection{Type annotations}
\begin{itemize}
\item \texttt{:} is used to annotate a variable or expression having a type.
\item \texttt{\_} can be used to mark a type wildcard (not a type variable, just something you don’t want to explicitly write). So you could have \texttt{[\_]} to constrain it to be a list of something.
\end{itemize}

\subsubsection{Rows and Unions}
Rows are composed of a series of key-type pairs or a union of two rows:

\begin{grammar}
<row body> ::= \{<ident> `:' <type> `,'\}
          \alt <ident>
          
<row union> ::= `'
           \alt <row_body> \{ `|' <row_body> \}

<row>      ::= `(' <row union> `)'
\end{grammar}

\todo[inline]{What syntax are we using for union types? Braces are taken for record syntax.}

\subsubsection{Records}
Record definitions are composed of a series of identifier keys to type pairs separated by commas within braces.

\begin{grammar}
<record>      ::= `\{' <row union> `\}'
\end{grammar}

\subsection{Type declarations}
Types can only be declared in a module. There are two forms of type declarations:

\begin{itemize}
\item Aliases: allowing referencing a longer type name as a shorter one (or renaming it if it conflicts or any other reason)
\item Type definitions: used for creating sum and product types.
\end{itemize}

\subsubsection{Type names}
Type names are defined as an identifier followed by a series of free type variables and/or concrete types.
The parameters that are specified within the RHS of the definition \textit{must} exist on the LHS.

\subsubsection{Type aliases}
The \texttt{type} keyword is used to define a type alias:

\begin{grammar}
<type name def> ::= <ident> \{ <var> \} 
               \alt `op' (`left' | `right') [<digit>] <var> <ident> <var>
               \alt <var> \lit{\`{}} <ident> \lit{\`{}} <var>

<type name>     ::= <type name>  [ `:' <type> ]

<type def> ::= `type' <type name> `=' <type>
          \alt `type' `foreign' <type name>
\end{grammar}

\subsubsection{Sum type definitions}
Sum types are composed of one or more type-constructors, each followed by the types the product type is composed of.

\begin{grammar}
<product name> ::= <ident> \{<type>\}
              \alt <type> `(' <ident> `)' <type>
              \alt <type> \lit{\`{}} <ident> \lit{\`{}} <type>

<product type> ::= <product name>  [ `:' <type> ]

<sum type>     ::= `data' <type name> `=' \{`|' <product type> \}
              \alt `data' `foreign' <sum name>
\end{grammar}

If the declaration starts on a new line it should have a leading `|', otherwise not.

\begin{minted}{haskell}
data X = A | B

data Y = 
    | C Int
    | D 
\end{minted}

\paragraph{GADTs}
GADTs are defined as so:

\begin{minted}{haskell}
data Foo a = Foo1 Int : Foo Int
           | Foo2     : Foo a
\end{minted}

Type variables defined on the RHS are accessible on the left:

\begin{minted}{haskell}
data Bar a = Bar1 a b : forall b . Foo a
           | Bar2     : Foo a
\end{minted}

\subsubsection{Type classes}
Type classes are marked with the \texttt{class} keyword. You can then specify a series of constraints on various parameters before specifying the actual type. All constraints’ parameters must appear in the type class’s definition.

\begin{grammar}
<class constraint> ::= <type> `=>'

<functional dep>   ::= \{ <ident> \} `->' \{ <ident> \}

<binding>  ::= <lhs> [`:' <type> ] [`=' <expr>]

<bindings> ::= <binding>
          \alt <binding> \{ `and' <binding> \}

<class body> ::= `type' <type name> [`=' <type> ]
            \alt `data' <type name> [`=' \{ '|' <product type> \} ]
            \alt `let' [`rec'] <bindings>

<class>            ::= `class' \{ <class constraint> \} <type> [ `|' <functional dep> \{ `,` <functional dep> \} ] `where' <class body>

<instance>         ::= `impl' <ident> \{ <var> \} `=' <type> `where' <class body>
\end{grammar}

\begin{minted}{haskell}
class Show a where -- or class Show a =
    let show : a -> String

impl X = Show String where
    let show x = "!" ++ x ++ "!"
\end{minted}

The beginning of the type class’s body is marked with the \texttt{where} keyword and an indent. The body contains a series of variables with type annotations with optional definitions. It is possible for all definitions to be filled and depend on one another: when creating an implementation Amulet will determine if you have provided sufficient information for a complete definition.

\paragraph{Type class implementations}
Implementations share a similar syntax to type class definitions, using \texttt{impl} instead of \texttt{class}. However, instead of following the type class with type parameters you can use any type expression.

The body of the implementation uses an identical syntax to that of type classes, though empty definitions are not allowed.