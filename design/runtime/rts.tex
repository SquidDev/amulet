\section{Run-time and Compile-time Semantics}

\subsection{The Runtime System}
Since Amulet is a pure language, there are many things that are fundamental to writing programs that could not be
implemented within it, the most important of which being I/O. Seeing as there is no way of expressing I/O actions in a
pure language, we expose untyped bindings from the runtime system, and have \textit{interface modules} link against
those.

The runtime system, or \textit{RTS} for short, provides, along with handle-based I/O,
\begin{itemize}
\item Automatic memory management (garbage collection) that can handle cyclic data structures.
\item An efficient linear data structure for high-performance code where linked lists are inappropriate (The
      $\mathtt{Vector}\ \alpha$ type).
\item Bindings to various big number frameworks (bigint, GMP, etc...), for arbitrary precision numbers.
\item Operations on unboxed numbers (integers and doubles).
\item Miscellaneous functions for management of thunks.
\item Support for infinite recursion.
\end{itemize}

The compiler knows about RTS functions, and will insert invocations to them where appropriate. Such applications are
marked with the compiler pragma \texttt{RTS}, and can only be present in modules not marked \texttt{Safe}, as they are
intrinsically unsafe.

Though \texttt{Safe} modules can't use RTS functions, there are bindings to all of the functions which may be considered
safe, such as \texttt{force\#} (which evaluates a thunk unconditionally), in safe modules. The binding for
\texttt{force\#} is $\mathtt{force}: \mathtt{Lazy}\ \alpha \to \alpha$.