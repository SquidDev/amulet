\subsection{Laziness}

Amulet, though strict by default, has standard library functions and compiler extensions to deal with laziness.

\subsubsection{Notation}

Before introducing the distinction between \texttt{Lazy} and \texttt{Strict} modules, we present a useful notation for
strictness analysis.

Strictness analysis mostly follows reduction rules that involve divergent terms, which are denoted by $\bot$. These such
terms will halt evaluation with an error message, as is the case with \texttt{error} or \texttt{undefined}, or reduce
into themselves, as is the case with the paradoxical $\omega$ combinator, $\left(\lambda{x}.x\ x\right)\
\left(\lambda{x}.x\ x\right)$. Though the $\omega$ combinator is not well-typed, similar looping terms are, such as
$\mathtt{let\ rec}\ x\ =\ x\ \mathtt{in} x$.

A function can be \textit{strict} in its argument. This is indicated by the following equivalence, which says that
\textit{$f$ applied to $\bot$ is $\bot$}.
\[
f\ \bot\ =\ \bot
\]

However, if a function is \textit{lazy} in its argument, when that function is applied to $\bot$, a concrete result is
returned. For example, given the function $\mathtt{f\ \tilde{x}} = 1$, then $f\ \bot \equiv 1$.

\subsubsection{Strict and Lazy Modules}

Amulet modules can be annotated through the use of compiler pragmas, such as the \texttt{LAZY} or \texttt{STRICT}
pragmas, which control strictness.

In a module that is strict by default (that is, every module without an annotation), then all lambda expressions are
automatically strict unless marked with a lazy pattern (this includes top-level combinators, as strictness is handled in
the desugared form). Formally,
\[
  \left(\lambda{p}. e\right)\ \bot \mapsto \bot\ \forall\ p\ \text{where \textit{p} is not lazy}
\]

In lazy modules, however, all patterns are automatically assumed lazy, unless explicitly marked strict (by use of a
bang-pattern, for example). This also has the consequence that any type variable $\alpha$ is converted to a lazy type,
such as $\mathtt{Lazy}\ \alpha$. This maintains the laziness properties across module boundaries, and has no
observable side effects, since $\mathtt{Lazy}\ \alpha \sim \alpha$.

\subsubsection{Thunks}

Thunks are the representation of a suspended - in other words, \textit{lazy} - computation. A thunk represents a value
that is yet to be evaluated. Since laziness is marked in a type (A lazy value doesn't have type $\alpha$, instead having
the type $\mathtt{Lazy}\ \alpha$), there is no added overhead checking if a value is a thunk.

The Amulet runtime takes advantage of the fact that thunks are uniformly represented as pointers to do \textit{thunk
  sharing}, which reduces duplication of work. While a na\"{i}ve compiler might generate two separate thunks for two
expressions that are semantically equivalent, the Amulet compiler will not. Since these thunks are shared, they'll only
be evaluated once.

As an example of thunk sharing, we present an Amulet combinator representing a computation to be executed twice, and how
that combinator would be simplified to share most computations.

\[
  \mathtt{let}\ f\ \_\ =\ \left(2 + 2\right) + \left(2 + 2\right)
\]

As the expression $\left(2 + 2\right)$ is used twice in exactly the same terms, it is simple to notice that this work
can be shared, by introducing a new, fresh let binding and giving that as the argument to the top-level
$\left(+\right)$.

\[
  \mathtt{let}\ f\ \_\
             = \mathtt{let}\ a = \left(2 + 2\right)
                \mathtt{in}\ a + a
\]

Given this definition, we can see how $\left(2 + 2\right)$ is now only computed once. Thunk sharing is used in Amulet
along with \textit{Thunk replacement}, which replaces a thunk for $x$ with the normal form of $x$.
