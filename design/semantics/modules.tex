\section{Modules}
Modules allow splitting components into separate components. Unlike Haskell, exports are defined at the definition level. By default variables are exported, though you can also specify other levels:

\begin{itemize}
\item Public: export this variable, allowing access anywhere.
\item Internal: export this variable, but only allowing access inside this compilation unit.
\item Private: do not export this method
\end{itemize}

It is also possible to re-export entries from other modules.

If a variable's signature consumes an object which is has a more restrictive export level then an error would occur. For instance:

\begin{minted}{fsharp}
data private X = X Int

(* An error occurs as f is public but X is private and so isn't exported *)
let public f (X a) : Int = a
\end{minted}

In the case of sum types, it is possible to export the type but not its constructors. This means it is trivial to provide a interface without exposing the implementation.

All definitions are accessible from any module when compilation occurs. This allows cross-module interaction and optimisation. However export checks occur after Amulet is parsed but before this exposing occurs so it is not possible to access inaccessible objects.

\subsubsection{Implicit modules}
Some constructs automatically create child modules which are imported into the primary module. These are generally used for types, and so you can index the type name to 

\paragraph{Type classes} All methods defined in a type class are automatically added to a submodule of the same name. This means you can use \texttt{Show.show} in addition to \texttt{Show}.

\paragraph{Tagged Unions} All constructors and destructors defined in a tagged union are added to a submodule of the same name. This means you can use \texttt{Expr.Var} as both a constructor and in pattern matching.